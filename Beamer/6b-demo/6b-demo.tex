% This text is proprietary.
% It's a part of presentation made by myself.
% It may not used commercial.
% The noncommercial use such as private and study is free
% Sep. 2005 
% Author: Sascha Frank 
% University Freiburg 
% www.informatik.uni-freiburg.de/~frank/


\documentclass{beamer}
\usecolortheme{whale}
\useoutertheme{split}
\usefonttheme[onlysmall]{structurebold}
\begin{document}
%\title{Simple Beamer Class}   
%\author{Sascha Frank} 
%\date{\today} 

%\frame{\titlepage} 

\frame{\frametitle{Table of contents}\tableofcontents} 

\section{Problem Statement}

\frame{

\frametitle{Does a 6 bit counter exists ?} 

Can we create a \textbf{deterministic} network that will iterate over all 6 bits string -- in any order -- ?

}

\frame{
\frametitle{Does a 6 bit counter exists ?} 
\center{
\includegraphics[scale=0.7]{netk.png}
}

}


\frame{\frametitle{Can we bruteforce that problem ?}
There are $\boldsymbol{2^{44}}$ differents networks which is approximatively $17 000$ billions.\\
We could hope for an answer in a few days.\\
\textbf{But} we can drastically restrict our search space with a \textbf{few} observations.
} 


\frame{\frametitle{How such a network's dynamic would look like ?} 

There are only two possibilities (up to any permutation):
\rotatebox{180}{
\includegraphics[scale=0.7]{2cases.png}
}


}

\frame{\frametitle{Implication on the computed function  by the network} 

The function that our counting network coumputes is either:
\begin{itemize}
\item \textbf{A bijection}
\item \textbf{An almost bijection} $f$ i.e, $\exists ! x_0,y_0 \in \{0,1\}^6$ such that $f'$ is a bijection where:  
\begin{align*}
\forall x \neq x_0 \quad & f'(x) = f(x) \\
& f'(x_0) = y_0 \\
\end{align*}
We are going to check on both cases.
\end{itemize} 
}

\section{Computable bijections on $\{0,1\}^6$}

\frame{\frametitle{The $4*16$ property} 

If we have a bijection it will \textbf{enumerate} all strings of  
$\{0,1\}^6$. After reordordering it will look like:\\ \ \\
\includegraphics[scale=0.6]{all6b.png}

}

\frame{\frametitle{The $4*16$ property} 

Let's focus on the two first bits, we can organise our sequences this way: \\ \ \\
\includegraphics[scale=0.55]{all6b2b.png}

We see that \textbf{each of the 2 bits pattern happens 16 times}.\\
It's the \textbf{$\boldsymbol{4*16}$ property}.

}

\frame{\frametitle{The $4*16$ property} 

\rotatebox{90}{
\includegraphics[scale=0.8]{subnetk.png}
}


}

\frame{\frametitle{Compute all the $4*16$ sub networks} 

By exausthive search we find $\boldsymbol{288}$ (over $16384$) sub networks with this property. \\ \ \\

By removing equivalent networks we are left with \textbf{$\boldsymbol{72}$ circuits for the 2 first bits}.

}

\frame{\frametitle{4*16 holds for middle and ending bits}
oo
}



\frame{\frametitle{How to conclude I ? Combine it !}

The work we have done for the 2 first bits can also be done for the 2 middle bits and the 2 last bits \textbf{that also have the 4x16 property}. \\ \ \\
We count:

\begin{itemize}
\item 72 networks for the 2 first bits
\item 216 networks for the 2 middle bits
\item 72 networks for the 2 last bits
\end{itemize}

\textbf{Hence} by combining these we have $72*216*72 \simeq 10^6$ 6 bits network to test.

} 

\frame{\frametitle{How to conclude II ? Count orbits !}

Over all these potential networks we count $\boldsymbol{497664}$ bijections. \\
For each of these bijections we have to \textbf{count their orbits}, we have a winner iif it has only 1 orbit. \\ \ \\
We do not find such a network, here there's the histogram of orbits: \\

\includegraphics[scale=0.5]{hist_orbits.png}

\textbf{Conclusion:} There are no bijective 6bits counters.


} 


\end{document}
